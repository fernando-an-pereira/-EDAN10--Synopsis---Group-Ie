\documentclass[a4paper]{article}
\usepackage[T1]{fontenc}
\usepackage[english]{babel}
\usepackage{amsmath}
\usepackage{graphicx}
\usepackage{url}
\usepackage{verbatim}

% Hypertext
\usepackage{hyperref}

%Bibliobraphy
\usepackage{natbib}
\usepackage{bibtopic}
\usepackage{url}

\usepackage[utf8]{inputenc} % Krävs för att svenska tecken ska läsas korrekt i vissa system.
%\usepackage[latin1]{inputenc} % Om svenska tecken inte fungerar korrekt, försök att byta ut föregående rad mot denna (eller testa utan någon av raderna)

%Allow the use of \verbatimtabinput which includes external files, and handling tabs correctly
\usepackage{moreverb}
\def\verbatimtabsize{4\relax} 

\bibliographystyle{unsrt}

%Remove red boxes due to the hyperref
\hypersetup{
    colorlinks,
    citecolor=black,
    filecolor=black,
    linkcolor=black,
    urlcolor=black
}
%%%%%%%%%%%%%%%%%%%%%%%%%%%%%

\title{EDAN10 -- Group I-e\\--\\ Synopsis \\
Introduction of SCM in a company 
%maybe add a better title
}

\author{Einar Holst \\
Fernando de Andrade Pereira \\
Hani Fakhouri
}

\begin{document}
\maketitle
\thispagestyle{empty}
\clearpage

\tableofcontents
\thispagestyle{empty}
\clearpage

\setcounter{page}{1}

\section{Keywords}
%introduce one keyword per line


\section{Topic}
In this work, we will discuss about the introduction of SCM in a company (specifically the \emph{PonteVecchio Software}).

The project can be considered quite relevant, due the increasing of productivity this corporation may have with the use of a SCM tool.

The main stakeholder of this project is company management, because the projects will be finished faster, so more products can be sold. 
As secondary stakeholders, the developers and testers of the company are identified, because they should agree with the changes in the company.
Others secondary stakeholders are costumers of this company, that can have the products in less time and with better quality.

\section{Problem statement}
A company (called \emph{PonteVecchio Software} or PVS for short) is a small software development company that has grown from 2 to 15 developers in the last 10 years. They have never introduced any form of configuration managment (CM for short) and are starting to notice that they have problems they can not quiet explain. They have heard that CM could help them with some of these problems and have requested a plan to introduce CM to the company in steps that will not disturbe their production but at the same time solve their issues.

Currently the company works on their source code using a single network resource to and from which everyone reads and writes. This is likely to be the cause of one of the issues they have been having which is that code that has been done seems to disappear from time to time without a trace. This is commonly known as the \emph{simultaneous update} problem in CM and is normaly solved using tools.

The simultaneous update problem that has been creeping into development has meant that the company has been forced to introduce a \emph{Waterfall method} of development that allows them dictate who can edit what files and when in order to minimize the problem. This in some ways might slow down production as delays in a single part of the plan set forward can cause chain reactions that delay the whole project.

Another issue that they have because of the shared network resource is that they do not have the ability to maintain old releases that have been delivered to costumers. While a copy of what was delivered to the costumer is kept on a CD which was created at release. While this copy does exist it does not provide any benefit to the company other then the ability to make a copy of it and deliver it to clients again if it is lost.

There is also an issue with that the company has no clear process for handling change requests which has resulted in some confusion among the employees. Mostly this happens when some interface or class changes how it is intended to be used or methods change name without the information that it will be changed or even has changed reaches all the developers.

While some of these issues could be easily fixed the company managment has also decided that they want to use exclusivly free tools to do so. While not a major problem it is one extra consideration to take into account when solving the issues the company has presently.

\section{Hypotheses}
Software configuration management must be considered in all software projects and companies as it eases the general wrokflow and increases maintenance of different project parts. Well defined SCM activities and standards should be conducted upon applying SCM. As a result a good structured SCM plan should be used and set at the begining in order to avoid unwanted problems in the future. A software company that does not use SCM might thus face many obstacles which results in more financial and technical costs and might decrease project performance and delay the delievery of company's product. 

This paper shall investigate and explain how SCM can be applied to a software company that does not use SCM at all. Main SCM parts that shall be introduced into the company are configuration identification, configuration change management and configuration auditing. The result of this work shall be a SCM plan that describes the standards that shall be used in the company. It is furthermore assumed that the company has never used SCM before but is willing to do so and the management is looking forward to apply different SCM practices. The company has encountered different management problems specially during development of its products and is aware of the severeness of such problems. These hard impacts made the company management decide to introduce SCM. 

\section{Method}

\newpage
\appendix
\section{Literature}

The literature that will be used in the project was prioritized in primary -- most relevant to this work -- and secondary -- less relevant, but also important in the work. 
\begin{btSect}[alpha]{primary}
\subsection{Primary}
\btPrintAll
\end{btSect}

\begin{btSect}[alpha]{secondary}
\subsection{Secondary}
\btPrintAll
\end{btSect}

\end{document}
